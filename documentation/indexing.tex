\subsection{Indexer package}
Индексация - процесс построения обратного индекса, то есть соответствия вида 
(токен - список документов, в которые входит этот токен). Это нужно для 
эффективного поиска документов, отвечающих поисковому запросу пользователя.
\par
Пакет Indexer обрабатывает список файлов, он парсит каждый файл из списка а далее заносит
изменения в базу данных. Индексирование также происходит конкурентно, список файлов делится 
на блоки фиксированного размера так, чтобы каждый блок файлов был примерно одинакового
размера. Далее каждой горутине передается блок файлов и она обрабатывает его с помощью 
пакета Parser. То есть внутри пакета Indexer создается общий sync.Map для индекса, а далее 
создается структура данных, описывающая обратный индекс.
\par
Обратный индекс формируется с помощью линейного перебора всех элементов индекса, затем документ 
добавляется в список  соответсвующего токена. 
\par
Также в базе данных для каждого файла сохраняется его индекс - соответсвие вида (токен - количество вхождений) - 
для последующего использования при поиске. 

\subsection{MongoDB, Models, Config packages}
После обработки блока файлов пакет Indexer вносит изменение в базу данных, 
это взаимодействие происходит благодаря пакетам MongoDB и Config. Их 
функционал заключается в обращении к базе данных и загрузке в нее 
обратного индекса. На вход пакет MongoDB принимает срез структур, описывающих 
обратный индекс, в них прописан сам токен и список документов, в которых содержится 
этот токен. 
\par
Функция, записыващая обратный индекс в базу данных, также проверяет, была ли создана 
коллекция, отвечающая определенному токену. Если нет, то она создает ее и с помощью 
работы с bson-файлами. 
\par
Для корректной работы с базой данных также предусмотрет пакет Config, 
который создает новый конфиг со всей нужной информацией для подключения - 
context, URI и прочее. 