\section{Существующие поисковые движки}
%Включает в себя обзор источников и раскрытие темы
Если мы говорим про крупнейшие поисковые движки, то в силу закрытого исходного кода, то мы не можем знать, 
какие именно алгоритмы токенизации и ранжирования они используют, однако существует масса решение с открытым 
исходным кодом. Среди самых используемых можно выделить Elasticsearch и Solr. Рассмотрим каждый из них и 
выделим, какие их части можно интегрировать в мой движок.

\subsection{Elasticsearch}
В первую очередь это поисковой движок для поиска по документам. Этот движок использует построение инвертированного 
индекса - индекса, который сопоставляет токенам соответствующие документы, в которых они встретились. Для этого используется 
библиотека Apache Lucene. Также в Elasticsearch может использоваться распределенный индекс - он хранится на нескольких 
серверах, что позволяет организовать горизонтальное масштабирование. Для эффективного поиска этот движок использует 
специальные метрики узлов - базы данных для хранения индекса - позволяющие распределять нагрузку по нескольким серверам. 

\subsection{Solr}
Начиная с 2010 года Solr и Lucene были объединены. Solr это Apache Lucene с дополнительным функционалом - поиском. 
Он использует всю ту же библиотеку Apache Lucene для построения инвертированного индекса, но добавляет алгоритм поиска 
и ранжирования результатов. 

\subsection{Вывод}
Из анализа существующих решений с открытым исходным кодом можно сделать вывод, что в моем поисковом движке следует 
также использовать инвертированных индекс, так как он позволяет осуществлять полнотекстовый поиск и сразу по токенам 
получать id релевантных документов. Это замедляет процесс индексирования, но кратно уменьшает время поиска.

Что касается взаимодействия с базой данных, то в связи с тем, что мой движок будет работать с одной физической базой данных, 
то я не буду разрабатывать алгоритмы, схожие с решениями Elasticsearch, однако безусловно стоит детально проработать
эффективное распределение нагрузки при большом количестве запросов.
\enlargethispage{\baselineskip}

% Пример ссылки на Рисунок \ref{fig:choosen_words_graph} и на Таблицу \ref{tab:table1} и на Листинг \ref{lst:listing1}.

% \subsection{Пример вставки и оформления рисунка}
% Текст

% \begin{figure}[h!]
% 	\centering
% 	% \includegraphics[scale=0.5]{images/select_sentences/choosen_words.png}
% 	\caption{График распределения сложности выбранных слов}
% 	\label{fig:choosen_words_graph}
% \end{figure}

% \subsection{Пример вставки и оформления кода}
% Пример того как ссылаться на Листинг \ref{lst:listing1}.

% \begin{lstlisting}[caption=Использование schedule в коде, label=lst:listing1]
% def sched_save():
%     schedule.every().hour.do(log_saver)
%     while True:
%         schedule.run_pending()
%         time.sleep(1)
% \end{lstlisting}

% \subsection{Пример вставки и оформления таблицы}
% Пример того как ссылаться на Таблицу \ref{tab:table1} по тексту.

% \begin{table}[h!]
%   \begin{center}
%     \caption{Your first table.}
%     \label{tab:table1}
%     \begin{tabular}{l|c|r} % <-- Alignments: 1st column left, 2nd middle and 3rd right, with vertical lines in between
%       \textbf{Value 1} & \textbf{Value 2} & \textbf{Value 3}\\
%       $\alpha$ & $\beta$ & $\gamma$ \\
%       \hline
%       1 & 1110.1 & a\\
%       2 & 10.1 & b\\
%       3 & 23.113231 & c\\
%     \end{tabular}
%   \end{center}
% \end{table}