\section{Заключение}
% В результате работы было создано приложение, позволяющее производить поиск по документам, 
% загружать их 

В ходе выполнения проекта были успешно реализованы все поставленные задачи, направленные на создание 
высокопроизводительного поискового движка. Основные достижения и функциональные возможности приложения включают:
\begin{itemize}
    \item Обработка и нормализация текстовых данных. Реализованы этапы токенизации, нормализации текста, 
    удаления стоп-слов, а также стемминга с использованием алгоритма 
    Портера для русского и английского языков.
    \item Построение инвертированного индекса. Разработана система индексирования, которая формирует обратный индекс, 
    сопоставляющий каждому токену список документов, где он встречается. Такой подход обеспечивает эффективный и 
    быстрый поиск по коллекции текстовых файлов.
    \item Реализация алгоритмов ранжирования. Для оценки релевантности документов к поисковому запросу была
    использована метрика TF-IDF. Это позволило проводить точное ранжирование результатов поиска на основе косинусного 
    сходства между вектором запроса и векторами документов.
    \item Интеграция с базой данных. Приложение взаимодействует с MongoDB, где хранится обратный 
    индекс, что обеспечивает устойчивость и масштабируемость системы.
    \item Разработан пользовательсткий интерфейсреализованный на Vue.js, он предоставляет 
    графическую оболочку для загрузки файлов, выполнения поисковых запросов и получения результатов, что значительно 
    упрощает взаимодействие с системой.
    \item Произведен деплой приложения с помощью nginx
\end{itemize}