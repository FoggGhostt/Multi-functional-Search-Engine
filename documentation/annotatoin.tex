\section*{Aннотация}
\addcontentsline{toc}{section}{Aннотация}

% Оказывается, что быстрый и эффективный поиск документов является актуальным для большого 
% количество людей. Зачастую нам необходимо отыскать в своих архивах, например, нужный нам .pdf файл, при этом под 
% рукой может не быть компьютера или другого привычного для нас устройства. Мой проект представляет 
% собой легковесное и многофункциональное решение для работы с документами. 

Данный проект представляет собой реализацию поискового движка, разработанного на языке Golang с использованием современной 
архитектуры и фронтенд-фреймворка Vue.js. Основной задачей работы является обеспечение эффективного поиска по документам 
посредством обработки текстовых данных с применением методов стемминга. Текстовые файлы разбиваются на токены, 
которые предварительно нормализуются, очищаются от стоп-слов и подвергаются обработке с помощью алгоритма 
Портера для русского и английского языков.

Полученные токены используются для построения инвертированного индекса, что позволяет быстро сопоставлять каждому токену список 
документов, в которых он встречается. Для оценки релевантности результатов поиска применяется метрика TF-IDF.
Ранжирование документов осуществляется по косинусному расстоянию между вектором поискового запроса и векторами
 документов. Важным аспектом реализации является использование конкурентных вычислений с помощью горутин, что значительно ускоряет
  процессы индексирования и обработки запросов.

Архитектура системы включает интеграцию серверной части, отвечающей за парсинг, индексацию и обработку поисковых запросов, с 
базой данных MongoDB для хранения обратного индекса, а также графическую оболочку на Vue.js для взаимодействия с 
пользователем. Дополнительно, протестированы ключевые модули приложения (Parser, Indexer, Search).