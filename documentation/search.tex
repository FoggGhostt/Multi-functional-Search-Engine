\subsection{Search package}
Ключевой аспект поискового движка - ранжирование результатов поиска. Данный пакет отвечает за упорядочивание 
документов по релевантности поисковому запросу. Приведу подробное описание реализованного алгоритма. 

Пакет Search обрабатывает поисковой запрос, формируя для него вектор TF-IDF. Это происходит в несколько этапов:
\begin{itemize}
    \item Нормализация запроса
    \item Разбиение на токены
    \item Удаление стоп слов
    \item Подсчет частоты употребления токена в запросе
    \item Подсчет обратной частоты токенов в коллекции документов
\end{itemize}

Подсчет частот происходит эффективно благодаря хранению в базе данных прямого и обратного индексов, что
позволяет быстро вычислять значения метрик. Аналилогично строится матрица TF-IDF для всех документов. 

Далее формируется список всех файлов коллекции. Затем происходит ранжирование результатов при помощи 
подсчета косинуса угла между вектором запроса и векторами документов:
\begin{align*}
    &\langle v, u \rangle_F = \sum_i v_iu_i
    &\|v\| = \sqrt{\langle v, v \rangle_F}&
    &\cos_i = \frac{\langle \mathrm{req_{vec}}, (\mathrm{doc_{vec}})_i\rangle_F}{\|\mathrm{req_{vec}\| \cdot \|(\mathrm{doc_{vec}})_i\|}}
\end{align*}

Таким образом, пакет Search отвечает за упорядочивание документов по релевантности в соответсвии с поисковым запросом. 

\subsection{Графическая оболочка}
Для написания пользовательского интерфейса был выбрал фреймворк Vue.js. Графическая оболочка позволяет 
загружать выбранные файлы на сервер, выполнять поисковые запросы и при необходимости скачивать
файлы, являющиеся результатами поиска. 

\subsection{Тестирование}
Для тестирования приложения были написаны тесты трех основных пакетов: Parser, Indexer и Search. 

\subsection{Деплой}
Произведен деплой приложения с помощью веб-сервера nginx. 