\section{Введение}
\subsection{Цель}
Цель работы заключается в разработке легковесного и многофункционального поискового движка для работы с различными данными. 

\subsection{Задачи}
\begin{itemize}
  \item{Исследовательская часть}
    \begin{itemize}
        \item {Изучить существующие поисковые движки, изучить их устройство и используемые технологии}
        \item {Изучить процесс токенизации, рассмотреть существующие библиотеки}
        %...
    \end{itemize}
  \item{Программная часть}
      \begin{itemize}
        \item{Разработать алгоритм токенизации, позволяющий за короткий временной промежуток обрабатывать большие объемы документов}
        \item {Разработать алгоритм для преобработки текста, который будет включать в себя такие части как токенизация, нормализация и стемминг}
        \item {Построить инвертированный индекс для последующего поиска}
        \item {Реализовать хранение индекса  в базе данных}
        \item {Разработать алгоритм ранжирования результатов поиска}
        \item{Реализовать графическую оболочку для вывода результатов поиска}
    \end{itemize}
\end{itemize}

\subsection{Актуальность работы}
На рынке представлено множество различных поисковых движков, однако большая часть отличается обширной кодовой базой и 
большими системными требованиями. Среди крупных игроков рынка можно выделить Google, Yandex, Baidu, Yahoo, Bing. 
Все эти компании разработали собственные поисковые движки, которые являются готовыми продуктами и ими пользуются 
ежедневно миллиарды человек. Однако их главная  проблема все еще в том, что их работа требует огромных производительных 
мощностей. Мой движок должен стать компромиссным решением, которое будет сочетать в себе многофункциональный поиск 
по нескольким типам файлов и обладать достаточной легковестностью. Таким образом можно считать, 
что актуальность работы продемонстрирована.

\subsection{План работы}
\begin{itemize}
    \item{Проанализировать существующие поисковые движки, изучить их \\
    устройство и используемые технологии}
    \item {Изучить процесс токенизации, рассмотреть существующие библиотеки}
    \item {Имплементировать алгоритм токенизации, позволяющий за короткий временной промежуток обрабатывать большие объемы документов }
    \item {Имплементировать веб-краулер для рекурсивного обхода веб-страниц и извлечения из них различных данных для индексирования}
    \item {Реализовать хранение индекса в базе данных, проработать ее эффективное взаимодействие с остальной программой}
    \item {Имплементировать поиск данных в индексе на основе поискового запроса пользователя }
    \item {Реализовать алгоритм ранжирования полученных результатов поиска}
    \item {Разработать графическую оболочку для просмотров результатов поиска}
  \end{itemize}